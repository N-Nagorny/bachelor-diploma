\anonsection{Введение}

Метод конечных разностей во временной области (англ.~Finite Difference Time Domain, FDTD) является одним из самых популярных и часто используемых на практике методов вычислительной электродинамики: он позволяет исследовать поведение системы сразу на широком диапазоне частот и обладает высоким потенциалом для параллелизации. Последнее обстоятельство делает особенно предпочтительным использование для проведения FDTD-симуляции не традиционных микропроцессоров с последовательной архитектурой, а специализированных высокопараллельных вычислительных устройств. Одними из наиболее доступных устройств такого рода являются видеокарты (англ.~Graphics Processing Unit, GPU).

Большинство доступных на рынке пакетов программ для электромагнитной симуляции включают в себя высокопроизводительные реализации FDTD для видеопроцессоров, однако имеющиеся проекты с открытым исходным кодом поддержки GPU не имеют. Проприетарные реализации часто недоступны из-за высокой стоимости соответствующих программных пакетов. Кроме того, применение закрытого программного обеспечения затрудняет изучение и модификацию непосредственно самого метода. Поэтому достаточно актуальной задачей является разработка свободной реализации метода, эффективно использующей ресурсы GPU для выполнения симуляции.

В рамках данной работы была разработана программа для FDTD-моделирования, способная эффективно использовать ресурсы GPU. При помощи этой программы была произведена тестовая симуляция распространения гармонического сигнала, излучаемого тонким симметричным вибратором в замкнутом счётном объёме. Также была разработана референсная CPU-реализация, и произведено сравнение производительности расчётов с использованием CPU и GPU.

\clearpage