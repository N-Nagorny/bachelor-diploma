\anonsection{Введение}

\textit{Метод конечных разностей во временной области} (англ.~\textit{Finite Difference Time Domain}, \textit{FDTD}) является одним из самых популярных и часто используемых на практике методов вычислительной электродинамики: он позволяет исследовать поведение системы сразу на широком диапазоне частот и обладает высоким потенциалом для параллелизации. Последнее обстоятельство делает особенно предпочтительным для проведения FDTD-симуляции использование не традиционных микропроцессоров с последовательной архитектурой, а специализированных высокопараллельных вычислительных устройств. Одними из наиболее доступных устройств такого рода 
являются \textit{графические процессорные устройства} (\textit{ГПУ}).

Большинство доступных на рынке пакетов программ для электромагнитной симуляции включают в себя высокопроизводительные реализации FDTD как для \textit{центральных процессорных устройств}~(\textit{ЦПУ}), так и для видеопроцессоров, однако имеющиеся проекты с открытым исходным кодом поддержки ГПУ не имеют. Проприетарные реализации часто недоступны из-за высокой стоимости соответствующих программных пакетов. Кроме того, применение закрытого программного обеспечения затрудняет изучение и непосредственную модификацию самого метода. Ввиду вышеперечисленных факторов достаточно актуальной задачей является разработка свободной реализации метода, эффективно использующей ресурсы ГПУ для выполнения симуляции.

% Гармонический сигнал; тонкий симметричный вибратор
В рамках данной работы была разработана программа, реализующая метод конечных разностей во временной области, способная эффективно использовать ресурсы ГПУ и позволяющая симулировать возбуждение электромагнитных колебаний при помощи точечных источников тока и напряжения. При помощи этой программы была произведена тестовая симуляция распространения гармонического сигнала, излучаемого тонким симметричным вибратором в замкнутом счётном объёме. Помимо этого, была 
разработана референсная ЦПУ-реализация, также было произведено сравнение производительности расчётов с использованием ЦПУ и ГПУ.

\clearpage