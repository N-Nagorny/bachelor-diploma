\section{Метод конечных разностей во временной области}

Метод конечных разностей во временной области относится к общему классу сеточных методов решения дифференциальных уравнений. В его рамках уравнения Максвелла подвергаются дискретизации с использованием центрально-разностной аппроксимации как по временной, так и по пространственным координатам. Полученные конечно-разностные уравнения решаются программными или аппаратными средствами в каждой точке временной сетки, причём компоненты вектора напряжённости магнитного поля смещены на половину шага дискретизации относительно компонент вектора напряженности электрического поля. Расчёт полей в ячейках сетки повторяется до тех пор, пока не будет получено решение поставленной задачи в интересующем промежутке времени.
Существует также большое количество расширений метода, наиболее популярным из которых являются: разнообразные поглощающие граничные условия, преобразование ближнего поля в дальнее, моделирование сосредоточенных активных и пассивных элементов. В данной работе из них были реализованы только базовый алгоритм FDTD и сосредоточенный резистивный источник напряжения.

\subsection{Принципы работы метода}

Рассмотрим систему из четырёх векторных уравнений Максвелла, записанных в системе единиц СИ:
\begin{equation*}
	\begin{cases}
		\Rot\vec{E} = \sigma_{H}\vec{H}-\parder{\vec{B}}{t}, \\
		\Rot\vec{H} = \sigma_{E}\vec{E}+\parder{\vec{D}}{t}, \\
		\Div\vec{D} = \rho, \\
		\Div\vec{B} = 0.
	\end{cases}
\end{equation*}

Здесь: \\
$ \vec{E} $ --- напряжённость электрического поля, \\
$ \vec{H} $ --- напряжённость магнитного поля, \\
$ \vec{B} $ --- магнитная индукция ($ \vec{B} = \mu\mu_{0}\vec{H} $), \\
$ \vec{D} $ --- электрическая индукция ($ \vec{D} = \varepsilon\varepsilon_{0}\vec{E} $), \\
$ \varepsilon $ --- относительная диэлектрическая проницаемость, \\
$ \mu $ --- относительная магнитная проницаемость, \\
$ \sigma_{E} $ --- удельная электрическая проводимость, \\
$ \sigma_{H} $ --- удельная магнитная проводимость, \\
$ \rho $ --- плотность стороннего электрического заряда, \\
$ \varepsilon_{0} $ --- электрическая постоянная ($ \varepsilon_{0} \approx 8,8542 $ $\text Ф/\text м$), \\
$ \mu_{0} $ --- магнитная постоянная ($ \mu_{0}= 4\pi \cdot 10^{-7} $ $\text {Гн}/\text м$).

Рассматривая уравнения Максвелла, можно заметить, что изменение значения вектора индукции электрического поля во времени (частная производная вектора $ \vec{D} $ по времени) зависит от изменения магнитного поля в пространстве (ротор вектора $ \vec{H} $). Поэтому значение вектора электрического поля в каждой точке пространства в определённый момент времени будет зависеть от значения этого же вектора в предыдущий момент времени и от изменения распределения вектора напряжённости магнитного поля в пространстве. Аналогичным образом значение вектора $ \vec{H} $ в определённой точке и в определённый момент времени зависит от своего значения в предыдущий момент времени и от изменения распределения вектора $ \vec{E} $  в пространстве.

Исходя из этих требований, на время выполнения каждой итерации алгоритма (расчёт значения вектора $ \vec{E} $ или $ \vec{H} $ в определённой точке в определённый момент времени) нам необходимо хранить в памяти компьютера значения векторов $ \vec{E} $ и $ \vec{H} $ в предыдущий момент времени.

Расчёт трёхмерных электромагнитных структур сильно усложняет вычисление ротора полей. В связи с этим американским математиком китайского происхождения Кейном И была разработана схема расчёта, в которой электрическая и магнитная сетки сдвинуты относительно друг друга так, что магнитное поле рассчитывается в точках, расположенных ровно между точками, в которых рассчитывается электрическое поле, и наоборот. Эта схема сейчас известна как сетка И. Её графическая модель представлена на рис. 1.

\subsection{Входные параметры задачи}

На практике для численного решения какой-либо задачи электродина-мики необходимо задать счётный объём. Счётный объём (или счётная область) --- это та область пространства, в пределах которой выполняется численное моделирование, и в пределах которой осуществляется непосредственное нахождение электромагнитных полей.

Счётная область разбивается на ячейки при помощи сетки И; в каждом узле сетки задаются значения электрических и магнитных проницаемостей и проводимостей. Чаще всего в качестве базового материала счётного объёма рассматривают вакуум (или воздух), в отдельных узлах сетки помещаются металлические или диэлектрические структуры. Тем не менее, алгоритм вполне позволяет задать произвольные значения вышеперечисленных величин для каждой точки объёма.

Кроме того, для моделирования реальных задач необходимы источники поля: некоторая структура, способная создавать возмущение внутри счётного объёма. Так, среди прочего, FDTD позволяет симулировать возбуждение электромагнитных колебаний при помощи падающей электромагнитной волны или точечного источника напряжения.

\subsection{Особенности метода}

Отличительной особенностью метода конечных разностей во временной области является его относительная простота. К достоинствам метода также можно отнести возможность создавать анимированные изображения распространения волновых процессов в счётном объеме, что может быть очень полезно для понимания того, что происходит с моделью, и позволяет удостовериться в её корректности.

Основной недостаток метода --- весь счётный объём должен быть разбит на ячейки сетки И, причём величина шага дискретизации по пространственным координатам должна быть достаточно малой по сравнению с наимень-шей длиной волны, встречающейся в конкретной задаче. Кроме того, эта величина ограничивает детализацию распределения материалов в пространстве, поэтому может оказаться, что счётный объём должен быть разделен на очень большое число ячеек, что означает большие затраты памяти и долгое моделирование.

Ещё одним недостатком FDTD является обязательность вычисления параметров поля в каждой точке счётного объёма. Так, при необходимости найти поле на некотором отдалении от источника придётся производить расчёт во всех точках, что находятся между источником и интересующей точкой.

К тому же, счётная область обязательно должна быть конечной. В большинстве случаев это достигается использованием искусственных граничных условий, но они, как правило, вызывают дополнительные искажения.

\subsection{Базовые уравнения}

Как уже было сказано, метод FDTD предполагает введение сетки, кото-рая на практике представляет собой обыкновенный трёхмерный массив, в котором хранятся векторы полей и пространственная структура. Расчёт заключается в том, что программа просматривает по очереди все элементы этого массива в порядке возрастания индексов, перевычисляя его элементы по приведенным ниже формулам.

Для компонентов магнитного поля:

\label{eq:BaseFdtdEquations}
\begin{align*}
	\Yee{H_x}{n+1/2}{i,j,k} &=
        \Yee{H_x}{n-1/2}{i,j,k} - \frac{\frac{\Delta{t}}{\mu}}
             {1+\frac{\sigma_H\Delta{t}}{2\mu}}
        \left[
            \frac{\yee{E_z}{n}{i,j+1,k} - \yee{E_z}{n}{i,j,k}}{\Delta{y}} -
            \frac{\yee{E_y}{n}{i,j,k+1} - \yee{E_y}{n}{i,j,k}}{\Delta{z}}
        \right], \\
	\Yee{H_y}{n+1/2}{i,j,k} &=
        \Yee{H_y}{n-1/2}{i,j,k} - \frac{\frac{\Delta{t}}{\mu}}
             {1+\frac{\sigma_H\Delta{t}}{2\mu}}
        \left[
            \frac{\yee{E_x}{n}{i,j,k+1} - \yee{E_x}{n}{i,j,k}}{\Delta{z}} -
            \frac{\yee{E_z}{n}{i+1,j,k} - \yee{E_z}{n}{i,j,k}}{\Delta{x}}
        \right], \\
	\Yee{H_z}{n+1/2}{i,j,k} &=
        \Yee{H_z}{n-1/2}{i,j,k} - \frac{\frac{\Delta{t}}{\mu}}
             {1+\frac{\sigma_H\Delta{t}}{2\mu}}
        \left[
            \frac{\yee{E_y}{n}{i+1,j,k} - \yee{E_y}{n}{i,j,k}}{\Delta{x}} -
            \frac{\yee{E_x}{n}{i,j+1,k} - \yee{E_x}{n}{i,j,k}}{\Delta{y}}
        \right],
\end{align*}

Для компонентов электрического поля:
\begin{align*}
	\fYee{E_x}{n+1}{i,j,k} &=
        \frac{1-\frac{\sigma_E\Delta{t}}{2\varepsilon}}
             {1+\frac{\sigma_E\Delta{t}}{2\varepsilon}} \fYee{E_x}{n}{i,j,k} +
        \frac{\frac{\Delta{t}}{\varepsilon}}
             {1+\frac{\sigma_E\Delta{t}}{2\varepsilon}}
        \left[
            \frac{\yee{H_z}{n+1/2}{i,j,k} - \yee{H_z}{n+1/2}{i,j-1,k}}{\Delta{y}} -
            \frac{\yee{H_y}{n+1/2}{i,j,k} - \yee{H_y}{n+1/2}{i,j,k-1}}{\Delta{z}}
        \right], \\
	\fYee{E_y}{n+1}{i,j,k} &=
        \frac{1-\frac{\sigma_E\Delta{t}}{2\varepsilon}}
	         {1+\frac{\sigma_E\Delta{t}}{2\varepsilon}} \fYee{E_y}{n}{i,j,k} +
        \frac{\frac{\Delta{t}}{\varepsilon}}
             {1+\frac{\sigma_E\Delta{t}}{2\varepsilon}}
        \left[
            \frac{\yee{H_x}{n+1/2}{i,j,k} - \yee{H_x}{n+1/2}{i,j,k-1}}{\Delta{z}} -
            \frac{\yee{H_z}{n+1/2}{i,j,k} - \yee{H_z}{n+1/2}{i-1,j,k}}{\Delta{x}}
        \right], \\
	\fYee{E_z}{n+1}{i,j,k} &=
        \frac{1-\frac{\sigma_E\Delta{t}}{2\varepsilon}}
             {1+\frac{\sigma_E\Delta{t}}{2\varepsilon}} \fYee{E_z}{n}{i,j,k} +
        \frac{\frac{\Delta{t}}{\varepsilon}}
             {1+\frac{\sigma_E\Delta{t}}{2\varepsilon}}
        \left[
            \frac{\yee{H_y}{n+1/2}{i,j,k} - \yee{H_y}{n+1/2}{i-1,j,k}}{\Delta{x}} -
            \frac{\yee{H_x}{n+1/2}{i,j,k} - \yee{H_x}{n+1/2}{i,j-1,k}}{\Delta{y}}
        \right].
\end{align*}

Выше приведены формулы, позволяющие вычислить каждую из компонент векторов напряжённости электрического и магнитного полей. В этих формулах используются следующие обозначения: \\
$ \sigma_E $ --- удельная электрическая проводимость материала в ячейке сетки; \\
$ \sigma_H $ --- удельная магнитная проводимость материала в ячейке сетки; \\
$ \varepsilon $ --- абсолютная диэлектрическая проницаемость материала; \\
$ \mu $ --- абсолютная магнитная проницаемость материала; \\
$ \Delta{t} $ --- шаг дискретизации по времени; \\
$ \Delta{x} $, $ \Delta{y} $, $ \Delta{z} $ --- шаги дискретизации по пространственным координатам. \\

Необходимо заметить, что величина $ \Delta{t} $ определяет частотные характеристики метода: наивысшая частота в спектре сигналов, распространение которых моделируется, не должна превышать $ f_{max} = \frac{1}{\Delta{t}} $.

Также величина $ \Delta{t} $ должна удовлетворять условию Куранта:
\begin{align*}
\Delta{t} < \frac{1}{c\sqrt{\frac{1}{\Delta{x}^2} + \frac{1}{\Delta{y}^2} + \frac{1}{\Delta{z}^2}}}
\end{align*}

Пересчёт значений компонент выполняется «на месте», то есть рассчи-танное в каждый последующий момент времени значение помещается в ту же ячейку сетки И, в которой находилось значение в предыдущий момент. Это позволяет несколько снизить требования к оперативной памяти, предъявляемые методом.

\subsection{Точечный резистивный источник}

Возможность моделировать сосредоточенные элементы вводится дополнительно, при этом границы применения метода существенно расширяются. В частности, точечный генератор напряжения с заданным внутренним сопротивлением оказывается весьма удобной моделью источника питания.

Достоинством этого дополнения FDTD является то, что вид уравнений меняется весьма незначительно. Если источник ориентирован вдоль оси Z, то в системе уравнений (1) изменится только одна составляющая второго векторного уравнения:

\begin{equation}
    \label{eq:LumpedSource:MaxwellEquationsAmendment}
    (\Rot\vec{H})_z = \varepsilon\frac{\partial{E_z}}{\partial{t}} +
        \sigma E_z + \frac{I_L}{\Delta{x}\Delta{y}},
\end{equation}

где $ I_L $ --- ток через источник.
После применения рассмотренной выше процедуры дискретизации с использованием сетки И, для соответствующей компоненты поля $ E_z $ получается простое уравнение:

\begin{multline}
    \label{eq:LumpedSource:FdtdEquationsAmendmentWithU}
    \fYee{E_z}{n+1}{i,j,k} =
        \fYee{C_E}{}{i,j,k}\fYee{E_z}{n}{i,j,k}~~+~~
        \Yee{C_H}{}{i,j,k}
        \left[
            \frac{\yee{H_y}{n+1}{i,j,k}-\yee{H_y}{n+1}{i-1,j,k}}{\Delta{x}}
        \right. - \\ -
        \left.
            \frac{\yee{H_x}{n+1}{i,j,k}-\yee{H_x}{n+1}{i,j-1,k}}{\Delta{y}} +
            \frac{\yee{V_s}{n+1}{}}{R\Delta{x}\Delta{y}}
        \right],
\end{multline}
где $C_E$ и~$C_H$ определяются по формулам:
\begin{equation}
    \newcommand\XA{\displaystyle
        \frac{\Delta{t}}{\yee{\epsilon}{}{i,j,k}}}
    \newcommand\XB{\displaystyle
        \frac{\Delta{t}\Delta{z}}{2R\yee{\epsilon}{}{i,j,k}\Delta{x}\Delta{y}}}
    % --
    \fYee{C_E}{}{i,j,k} = \frac{1-\XB}{1+\XB}, \quad
    \fYee{C_H}{}{i,j,k} = \frac{\XA}{1+\XB}.
\end{equation}

Формулы для прочих компонент при этом остаются без изменения.