\section{Метод конечных разностей во временной области}

Метод конечных разностей во временной области относится к общему классу сеточных методов решения дифференциальных уравнений. В его рамках уравнения Максвелла подвергаются дискретизации с использованием центрально-разностной аппроксимации как по временной, так и по пространственным координатам. Полученные конечно-разностные уравнения решаются программными или аппаратными средствами в каждой точке временной сетки, причём компоненты вектора напряжённости магнитного поля смещены на половину шага дискретизации относительно компонент вектора напряженности электрического поля. Расчёт полей в ячейках сетки повторяется до тех пор, пока не будет получено решение поставленной задачи в интересующем промежутке времени.
Существует также большое количество расширений метода, наиболее популярным из которых являются: разнообразные поглощающие граничные условия, преобразование ближнего поля в дальнее, моделирование сосредоточенных активных и пассивных элементов. В данной работе из них были реализованы только базовый алгоритм FDTD и сосредоточенный резистивный источник напряжения.

\subsection{Принципы работы метода}

Рассмотрим систему из четырёх векторных уравнений Максвелла, записанных в системе единиц СИ:
\begin{align*}
	\Rot\vec{E} &=\sigma_{H}\vec{H}-\parder{\vec{B}}{t}, \\
	\Rot\vec{H}=\sigma_{E}\vec{E}+\parder{\vec{D}}{t}
\end{align*}
