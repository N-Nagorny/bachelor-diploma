\usepackage[utf8]{inputenc}
\usepackage[T1]{fontenc}

\usepackage{subcaption}
\usepackage{floatrow}
\usepackage{fontspec} % XeTeX
\usepackage{xunicode} % Unicode для XeTeX
\usepackage{xltxtra}  % Верхние и нижние индексы
\usepackage{pdfpages} % Вставка PDF

% Шрифты, xelatex
\defaultfontfeatures{Ligatures=TeX}
\setmainfont{Times New Roman}
\newfontfamily\cyrillicfont{Times New Roman}
\newfontfamily\cyrillicfonttt{FreeMono}
\setmonofont{FreeMono} % Моноширинный шрифт для оформления кода

% Русский язык
\usepackage{polyglossia}
\setdefaultlanguage{russian}

\usepackage{amssymb,amsfonts,amsmath} % Математика
\numberwithin{equation}{section} % Нумерация формул

% Reference style
\usepackage[hidelinks]{hyperref}
\urlstyle{same}

\usepackage{tabularx}
\usepackage{enumerate} % Тонкая настройка списков
\usepackage{indentfirst} % Красная строка после заголовка

\renewcommand{\baselinestretch}{1.5} % Полуторный междустрочный интервал
\parindent 1.25cm % Абзацный отступ


\sloppy             % Избавляемся от переполнений
\hyphenpenalty=1000 % Частота переносов
\clubpenalty=10000  % Запрещаем разрыв страницы после первой строки абзаца
\widowpenalty=10000 % Запрещаем разрыв страницы после последней строки абзаца

% Отступы у страниц
\usepackage{geometry}
\geometry{left=3cm}
\geometry{right=1.5cm}
\geometry{top=2cm}
\geometry{bottom=2cm}

\usepackage{lastpage}

% Move page numbers up
\usepackage{fancyhdr}

\fancyhf{}
\chead{\thepage}
\renewcommand{\headrulewidth}{0pt}
\pagestyle{fancy}

% Списки
\usepackage{enumitem}
\setlist[enumerate,itemize]{leftmargin=12.7mm} % Отступы в списках

\makeatletter
    \AddEnumerateCounter{\asbuk}{\@asbuk}{м)}
\makeatother
\setlist{nolistsep} % Нет отступов между пунктами списка
\renewcommand{\labelitemi}{--} % Маркер списка
\renewcommand{\labelenumi}{\asbuk{enumi})} % Список второго уровня
\renewcommand{\labelenumii}{\arabic{enumii})} % Список третьего уровня

\newcommand\centerThe[1]{\noindent\begin{center}#1\end{center}\par}

% Содержание
\usepackage{textcase}
\usepackage[subfigure]{tocloft}
\AtBeginDocument{\addtocontents{toc}{\protect\thispagestyle{fancy}}} 
\renewcommand{\contentsname}{Содержание}
\renewcommand{\cfttoctitlefont}{\bfseries\centerThe}
\renewcommand{\cftaftertoctitle}{}
\renewcommand{\cftsecfont}{\hspace{0pt}}            % Имена секций в содержании не жирным шрифтом
\renewcommand\cftsecleader{\cftdotfill{\cftdotsep}} % Точки для секций в содержании
\renewcommand\cftsecpagefont{\mdseries}             % Номера страниц не жирные
\setcounter{tocdepth}{3}                            % Глубина оглавления, до subsubsection

%Формулы
\DeclareMathOperator{\Div}{div}
\DeclareMathOperator{\Rot}{rot}
\newcommand{\parder}[2]{\frac{\partial {#1}}{\partial {#2}}}

\newcommand\yee[3]{\left.#1\right|^{#2}_{#3}}
\newcommand\Yee[3]{#1\Big|^{#2}_{#3}}
\newcommand{\hanglimitsoperator}[4][c]{\text{\makebox[\widthof{$#2$}][#1]{ $#2^{#3}_{#4}$ }}}

\newcommand\fyee[3]{\hanglimitsoperator[l]{\left.#1\right|}{#2}{#3}~~~}
\newcommand\fYee[3]{\hanglimitsoperator[l]{#1\Big|}{#2}{#3}~~~}
\newcommand\yeediff[5]{\frac{\yee{#1}{#2}{#3} - \yee{#1}{#2}{#4}}{\Delta{#5}}}

% Точка в заголовках рисунков и таблиц
\addto\captionsrussian{\renewcommand{\figurename}{Рисунок}}
\usepackage[labelsep=endash]{caption}
\numberwithin{figure}{section}

% Roman numerals
\makeatletter
\newcommand*{\rom}[1]{\expandafter\@slowromancap\romannumeral #1@}
\makeatother

% Code listings
\usepackage{listings}
\renewcommand{\lstlistingname}{Листинг}
\lstset{
    basicstyle=\small\ttfamily, % Размер и тип шрифта
    breaklines=true, % Перенос строк
    tabsize=2, % Размер табуляции
    literate={--}{{-{}-}}2 % Корректно отображать двойной дефис
}

% Bibliography items marking
\makeatletter 
\renewcommand\@biblabel[1]{#1.} 
\makeatother

% Оформление заголовков
\usepackage[compact,explicit]{titlesec}
\usepackage{titletoc} %Точки после номеров секций в содержании
\titleformat{\section}[block]{}{}{12.5mm}{\bfseries\thesection\quad#1}
\titleformat{\subsection}[block]{}{}{12.5mm}{\bfseries\thesubsection\quad#1}
\titleformat{\subsubsection}[block]{\vspace{1em}\normalsize}{}{12.5mm}{\thesubsubsection\quad#1\vspace{1em}}
\titlespacing*{\subsection}{0pt}{5.5ex plus 1ex minus .2ex}{0pt}
\titleformat{\paragraph}{}{}{12.5mm}{#1}
\titlespacing{\paragraph}{12.5mm}{*4}{*1.5}

% Anonymous section
\newcommand{\anonsection}[1]{
    \phantomsection % Корректный переход по ссылкам в содержании
    \paragraph{\bfseries{#1}}
    \addcontentsline{toc}{section}{#1}
}

%Appendix section
\newcommand{\appendsection}[1]{
    \phantomsection % Корректный переход по ссылкам в содержании
    \paragraph{\hfill\bfseries{#1}}
    \addcontentsline{toc}{section}{#1}
}