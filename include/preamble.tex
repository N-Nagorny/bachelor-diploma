\usepackage[utf8]{inputenc}
\usepackage[T1]{fontenc}

\usepackage{fontspec} % XeTeX
\usepackage{xunicode} % Unicode для XeTeX
\usepackage{xltxtra}  % Верхние и нижние индексы
\usepackage{pdfpages} % Вставка PDF

% Шрифты, xelatex
\defaultfontfeatures{Ligatures=TeX}
\setmainfont{Times New Roman}
\newfontfamily\cyrillicfont{Times New Roman}
\newfontfamily\cyrillicfonttt{FreeMono}
\setmonofont{FreeMono} % Моноширинный шрифт для оформления кода

% Русский язык
\usepackage{polyglossia}
\setdefaultlanguage{russian}

\usepackage{amssymb,amsfonts,amsmath} % Математика
\numberwithin{equation}{section} % Нумерация формул

\usepackage{tabularx}
\usepackage{enumerate} % Тонкая настройка списков
\usepackage{indentfirst} % Красная строка после заголовка

\renewcommand{\baselinestretch}{1.5} % Полуторный междустрочный интервал
\parindent 1.25cm % Абзацный отступ

\sloppy             % Избавляемся от переполнений
\hyphenpenalty=1000 % Частота переносов
\clubpenalty=10000  % Запрещаем разрыв страницы после первой строки абзаца
\widowpenalty=10000 % Запрещаем разрыв страницы после последней строки абзаца

% Отступы у страниц
\usepackage{geometry}
\geometry{left=3cm}
\geometry{right=1cm}
\geometry{top=2cm}
\geometry{bottom=2cm}

% Списки
\usepackage{enumitem}
\setlist[enumerate,itemize]{leftmargin=12.7mm} % Отступы в списках

\makeatletter
    \AddEnumerateCounter{\asbuk}{\@asbuk}{м)}
\makeatother
\setlist{nolistsep} % Нет отступов между пунктами списка
\renewcommand{\labelitemi}{--} % Маркер списка
\renewcommand{\labelenumi}{\asbuk{enumi})} % Список второго уровня
\renewcommand{\labelenumii}{\arabic{enumii})} % Список третьего уровня

% Содержание
\usepackage{textcase}
\usepackage{tocloft}
\renewcommand{\cftsecfont}{\hspace{0pt}}            % Имена секций в содержании не жирным шрифтом
\renewcommand\cftsecleader{\cftdotfill{\cftdotsep}} % Точки для секций в содержании
\renewcommand\cftsecpagefont{\mdseries}             % Номера страниц не жирные
\setcounter{tocdepth}{3}                            % Глубина оглавления, до subsubsection

%Формулы
\DeclareMathOperator{\Div}{div}
\DeclareMathOperator{\Rot}{rot}
\newcommand{\parder}[2]{\frac{\partial {#1}}{\partial {#2}}}

\newcommand\yee[3]{\left.#1\right|^{#2}_{#3}}
\newcommand\Yee[3]{#1\Big|^{#2}_{#3}}
\newcommand{\hanglimitsoperator}[4][c]{\text{\makebox[\widthof{$#2$}][#1]{ $#2^{#3}_{#4}$ }}}

\newcommand\fyee[3]{\hanglimitsoperator[l]{\left.#1\right|}{#2}{#3}~~~}
\newcommand\fYee[3]{\hanglimitsoperator[l]{#1\Big|}{#2}{#3}~~~}
\newcommand\yeediff[5]{\frac{\yee{#1}{#2}{#3} - \yee{#1}{#2}{#4}}{\Delta{#5}}}

% Оформление заголовков
\usepackage[compact,explicit]{titlesec}
\usepackage[dotinlabels]{titletoc} %Точки после номеров секций в содержании
\titleformat{\section}{}{}{12.5mm}{\centering{\thesection.\quad#1}\vspace{1.5em}}
\titleformat{\subsection}[block]{\vspace{1em}}{}{12.5mm}{\thesubsection.\quad#1\vspace{1em}}
\titleformat{\subsubsection}[block]{\vspace{1em}\normalsize}{}{12.5mm}{\thesubsubsection\quad#1\vspace{1em}}
\titleformat{\paragraph}[block]{\normalsize}{}{12.5mm}{#1}

% Секции без номеров (введение, заключение...), вместо section*{}
\newcommand{\anonsection}[1]{
    \phantomsection % Корректный переход по ссылкам в содержании
    \paragraph{\centerline{{#1}}\vspace{1.5em}}
    \addcontentsline{toc}{section}{#1}
}