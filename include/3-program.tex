\section{Программная реализация}

В рамках курсовой работы был разработан прототип программного обеспечения, рассчитывающего электродинамические структуры методом конечных разностей во временной области с использованием графического процессора. Применение ГПУ было призвано улучшить производительность вычислений.

\subsection{Программная реализация сетки И}

Разработка программы велась на языке С++, поэтому в качестве программного представления сетки И было решено использовать тип данных «класс». Входными аргументами конструктора класса выступают линейные размеры сетки в ячейках и величины $ \Delta{x} $, $ \Delta{y} $, $ \Delta{z} $, $ \Delta{t} $.

Из рис.~\ref{fig:YeeGrid} легко заметить, что все данные сетки (значения диэлектрической и магнитной проницаемости, удельной электрической и магнитной проводимостей, проекций векторов электрической и магнитной напряжённостей поля на координатные оси) представляют собой трёхмерные массивы с номерами ячеек $ i $, $ j $, $ k $ в качестве индексов.

\subsection{Базовый алгоритм FDTD}

Первым этапом создания программной реализации метода конечных разностей во временной области стало выделение из формул базового алгоритма коэффициентов и написание функций для их расчёта. Неизменность коэффициентов позволяет рассчитывать их единожды при старте программы и использовать готовые значения при пересчёте характеристик поля в каждой ячейке в каждый момент времени.

\begin{align}
\label{eq:CDCoeffs}
C=\frac{\Delta{t}}{Ƥ}
\end{align}

Данные коэффициенты приведены в формуле~\eqref{eq:CDCoeffs}, где Ƥ --- проницаемость материала (диэлектрическая и магнитная для проекций векторов $\vec{E}$  и $\vec{H}$ соответственно), а $\sigma$ --- удельная проводимость (электрическая и магнитная для проекций векторов $\vec{E}$  и $\vec{H}$ соответственно).

Следующим шагом стало написание функций, предназначенных для расчё-
та проекций векторов электрической и магнитной напряжённости во всех точ-
ках счётного объёма в какой-либо момент времени. Программный код, произ-
водящий расчёт компонент вектора магнитной напряжённости, приведён в лис-
тинге 1.

\subsection{Тестовая задача}

Симметричный вибратор --- простейшая система для получения электромагнитных колебаний. Представляет собой электрический диполь, дипольный момент которого быстро изменяется во времени, и является развёрнутым колебательным контуром с минимальной ёмкостью и индуктивностью [12].

В исходном коде конечной программы симметричный вибратор был пред-
ставлен двадцатью ячейками счётного объёма, расположенными вдоль оси , с
отличными от остальных удельной электрической проводимостью материала
и абсолютной диэлектрической проницаемостью материала . Длина волны бы-
ла подобрана таким образом, чтобы отношение длины диполя к длине волны
было равно двум.

\subsection{Реализация резистивного источника}

Следующим этапом моделирования резистивного источника стал пересчёт
значений проекции вектора ⃗ на ось . Во избежание проверки каждой ячейки
на наличие там проводящих структур было принято решение сохранять инфор-
мацию обо всех ячейках до расчёта значений вектора электрической напряжён-
ности, затем выполнять этот расчёт и, наконец, пересчёт значений проекции
вектора ⃗ на ось
только в точках присутствия элементов источника.

\subsection{Перенос вычислительной нагрузки на графический процессор}

В целях увеличения производительности функции расчёта значений про-
екций векторов электрической и магнитной напряжённостей были изменены
для исполнения на графических процессорах. Из двух технологий, позволяю-
щих осуществить расчёт на GPU: NVIDIA CUDA и OpenCL — была выбрана
вторая, так как она поддерживает процессоры не только производства компа-
нии NVIDIA, но и AMD.
Для облегчения работы с фреймворком OpenCL была использована биб-
лиотека EasyCL [13]. Код расчёта компонент вектора магнитной напряжённости
приведён на листинге 2.

\clearpage